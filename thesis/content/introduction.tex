\chapter{Introduction}

This chapter aims at giving an introduction and overview of the thesis. It starts with a brief explanation of why Internet today still feels slow despite major advances in technology, followed up by establishing the goals and research questions for the thesis. To address the research questions, a small look into the research methodology is presented. In the final section, an outline of the thesis structure is given.









\section{Background and Motivation}

The Internet has evolved beyond its original purpose to send mail or use the \gls{www}. New expectations from end users are rapidly increasing by the demand for interactive applications such as online gaming, audio or video streaming. A common ground for these services is that they are sensitive to latency. Despite the growing use of such services, the Internet is still suffering from poor latency and performance. The culprit is widely known as \textit{bufferbloat} \cite{bufferbloat}, the existence of excessively large and frequently full buffers inside the network causing the dreaded notion of \textit{lag}.

A key factor for the frequently full buffers is the dominant use of the \gls{tcp} as the main communication protocol for the Internet. Although \gls{tcp} itself is not the problem, a part of it became a major contributor. One of the main reasons for \gls{tcp} dominance is \gls{cc} --- a fundamental set of mechanisms for maintaining the stability and efficiency of the Internet. \cite{rfc6077} However, most mechanisms in use today still rely on lost packets as feedback for moderating the transmission rate of a host. This has served to add fuel to the bufferbloat problem, as standard \gls{cc} will fill up any buffer in the network until a packet loss is inferred. Constantly pushing excessively large buffers to its limits causes packets to become queued for long periods of time, resulting in high latencies.

To combat bufferbloat, the proposal of \gls{aqm} on intermediary devices was introduced in order to minimize the time packets spend enqueued at a bottleneck. By dropping a packet inside a buffer before it becomes full, the \gls{cc} would trigger a rate reduction before pushing the buffer to its limits. Building upon \gls{aqm} came the ability to explicitly signal congestion in the network, called \gls{ecn}. Although \gls{ecn} was standardized two decades ago, it has not seen much support due to compatibility issues with existing network equipment. However, recent years have shown a rapid change on this matter as the adoption of \gls{ecn} on end-systems has accelerated and is now supported on the majority of servers. \cite{enabling_internet-wide_ecn}

The benefits of \gls{ecn} are evident. Instead of reacting to network congestion after the fact (i.e. after packet loss), one can take protective \remark{?} measures before it happens. When reliable delivery is necessary, packet loss is directly responsible for reducing throughput and consequently rising the latency due to the additional time needed for retransmission. While the reason for a packet loss can be of multiple sources besides a full buffer, explicit feedback using \gls{ecn} serves as a clear signal for the inevitable impending of network congestion.

\gls{abe} \cite{abe} is a recent proposal \remark{technical paper?} that clearly shows the potential \remark{practical?} benefits of using explicit congestion feedback --- by reducing the sender's transmission rate upon the receipt of an \gls{ecn} mark, a packet loss can be avoided. In addition, the earlier congestion feedback allows for a less aggressive reduction factor, yielding a higher sustained throughput and lower latency while also maintaining the benefit of avoiding packet loss.









\section{Goals and Research Questions} \label{goals_and_research_questions}

In this section, the goal and research question for the thesis is presented along with a brief motivation.

\begin{statement}[Goal statement:]
    Designing and implementing an improved version of \gls{abe} on FreeBSD.
\end{statement}

The goal of the thesis is to improve \gls{abe} by making it more responsive to different network conditions. Currently, \gls{abe} shares the same behaviour as NewReno when responding to a congestion signal, which is to reduce a sender's transmission rate by a constant factor. The only difference is that \gls{abe} reduces by less. \todo{rework?} Hence, the following research question defined below will be investigated.

\begin{statement}[Research question:]
    \gls{abe} reduces a sender's transmission rate by a constant factor in the event of an \gls{ecn} congestion signal. Can this factor be made more dynamic such that it reduces less when there is moderate congestion, and more when there is heavy congestion?
\end{statement}

Since \gls{abe} backs off by a constant factor, the thesis will look into techniques to make this reduction factor more dynamic. The \gls{ecn} congestion response should back off less when there is moderate congestion, and back off more when there is heavy congestion. \todo{more?}









\section{Research Methodology}

To address the research question outlined above, a cluster consisting of eight Raspberry Pi machines has been set up. The cluster serves as a physical testbed for conducting various \gls{tcp} experiments in order to validate the results from our modifications to \gls{abe}. \todo{more....}









\section{Contributions}

\todo{write when finished}







\section{Thesis Structure}

The thesis consists of five chapters. This chapter presents the introduction and motivation for the research question. Chapter \ref{ch:background} gives an insight into the background knowledge that the thesis builds upon, and serves as an aid for those unfamiliar with the topic. Chapter \ref{ch:methodology} describes how the physical testbed has been set up, how our work has been done and the implementation details of it. Chapter \ref{ch:experimental_evaluation} presents the results from our work along with an evaluation of the findings. Chapter \ref{ch:conclusion} concludes the thesis with an outline of future considerations and a final section discussing the struggles we encountered during the project.