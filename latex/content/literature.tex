\chapter{Literature Review}

This chapter presents the background theory for which this thesis is based upon.


\section{Network Delay and Latency}

The time it takes for a bit of data to travel across the network from one communication endpoint to another is known as \textit{delay}. The process for such a transmission involves many components. Let us demonstrate with a typical example where the data traverses an intermediate device before reaching destination. First, the data to be sent is usually created by an application. The data will then be handed over to the \gls{os} which passes it to a network card. From there, the data will be encoded and transmitted over a physical medium and eventually received by an intermediate device, such as a router. The router will then analyze the data and retransmit it over another medium that points to the destination. Finally, the data reaches the receiver. The whole process can happen in either multiples or fractions of seconds.

Network delay is therefore divided into the following four parts:

\begin{itemize}
    \item Processing delay – time it takes router to process the packet header
    \item Queuing delay – time the packet spends in routing queues
    \item Transmission delay – time it takes to push the packet's bits onto the link
    \item Propagation delay – time for a signal to reach its destination
\end{itemize}

It is common to notify the sender that the receiver actually got the data. This is done by sending a signal from the receiver to the sender, known as an \gls{ack}. The total time it takes for a sender to send data \textit{and} receive back an \gls{ack} is known as \textit{latency} or \textit{rtt}.

In this thesis, we are mainly concerned with the \textit{queuing} delay part.

\section{Bufferbloat}

A common cause of latency in packet switched networks is bufferbloat. It occurs when a router, with a large buffer gets congested. the tcp \gls{cc} will fill up the entire buffer, before it starts backing off. Packets become queued for a long period of time, untill the buffer is drained, \gls{cc} resets and TCP connection ramps back up to speed to fill the buffer again.

This causes high and variable latency, in addition to "blocking" the bottleneck for other flows when the buffer is full and packets are droped. Several technical solutions exists, that try to solve the problem of bufferbloat and we will outline some of them  in this section.

\section{Network Congestion}

\todo{write about congestion...}

\section{Transmission Control Protocol}

\gls{tcp} is the one of the main protocols for transmitting data on the internet. It is a connection based, reliable protocoll and is used by for instance World Wide Web (WWW), email, File Transfer Protocoll (FTP) and streaming media. \gls{tcp} requires that the sender and reciever establishes a connection through a three-way handshake before transmission starts. All segments sent have a sequence number, and the reciver sends an \gls{ack} for every segment it recieves. This, in addition to retransmission and error-checking ensures reliable transfer, but also lengthens latency.

\subsection{Congestion Control}

To help reduce congestion on the links \gls{tcp} maintains a \gls{cwnd}, which limits the total number of unacknowledged packets that it can send at a time. This is  done in multiple fases.

In the slow start fase,  right after a connection is established. the congestion window starts as a small multiple of \gls{mss} and is effectivley doubled for every \gls{rtt}. When it reaches the slow-start threshold(ssthresh), \gls{cwnd} is reduced by half and a new fase starts, congestion avoidance. In this fase \gls{cwnd} is increased linearly by one \gls{mss} every \gls{rtt}. If loss occurs, it could mean there is congestion, and steps will be taken to reduce load on the network. The steps depend on what exact congestion avoidance algorithm is used.

\section{Active Queue Management}

\section{Explicit Congestion Notification}

\gls{ecn} is an extension to TCP that allows routers to notify end points on impending congestion without dropping packets.

\subsection{Legacy ECN}
In legacy ECN, the router notifies end hosts of congestion by setting a Congestion Encountered (CE) flag in the IP header on ECN enabled packets when experiencing congestion. The reciever of the packet then reflects this back to the sender by setting an ECN-Echo (ECE) in the TCP header. It keeps doing this until the sender responds back with a segment with Congestion Window Reduced (CWR) set, indicating that the sender has backed off.


\section{Alternative Backoff with ECN}

