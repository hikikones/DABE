\chapter{Literature Review}

This chapter presents the background theory for which this thesis is based upon.


\section{Network Delay and Bufferbloat}
A common cause of latency in packet switched networks is bufferbloat. It occurs when a router, with a 
large buffer gets congested. the tcp congestion controll will fill up the entire buffer, before 
it starts backing off. Packets become queued for a long period of time, untill the buffer is drained,
congestion controll resets and TCP connection ramps back up to speed to fill the buffer again.
This causes high and variable latency, in addition to "blocking" the bottleneck for other flows
when the buffer is full and packets are droped. 
Several technical solutions exists, that try to solve the problem of bufferbloat and we will outline 
some of them  in this section.
\section{Transmission Control Protocol}
Transmission Controll Protocol (TCP) is the one of the main protocols for transmitting data on the internet.
It is a connection based, reliable protocoll and is used by for instance World Wide Web(WWW), email, 
File Transfer Protocoll (FTP) and streaming media. TCP requires that the sender and reciever establishes a connection 
through a three-way handshake before transmission starts. All segments sent have a sequence number, and the reciver sends 
an acknowledgment(ACK) for every segment it recieves. This, in addition to retransmission and error-checking ensures
reliable transfer, but also lengthens latency.

\subsection{Congestion Control}
To help reduce congestion on the links TCP maintains a congestion window(CWIN), which limits the total number of 
unacknowledged packets that it can send at a time. This is  done in multiple fases.
In the slow start fase,  right after a connection is established. the congestion window starts as a small multiple 
of MSS(Maximum Segment Size) and is effectivley doubled for every RTT. When it reaches the slow-start threshold(ssthresh),
CWIN is reduced by half and a new fase starts, congestion avoidance. In this fase CWIN is increased linearly by one MSS 
every RTT. If loss occurs, it could mean there is congestion, and steps will be taken to reduce load on the
network. The steps depend on what exact congestion avoidance algorithm is used.

\section{Active Queue Management}

\section{Explicit Congestion Notification}
Explicit Congestion Notification (ECN) is an extension to TCP that allows routers to notify end points on impending congestion
without dropping packets.

\subsection{Legacy ECN}
In legacy ECN, the router notifies end hosts of congestion by setting a Congestion Encountered (CE) flag in the IP header on ECN enabled packets
when experiencing congestion. The reciever of the packet then reflects this back to the sender by setting an ECN-Echo (ECE) in the TCP
header. It keeps doing this until the sender responds back with a segment with Congestion Window Reduced (CWR) set,
indicating that the sender has backed off.


\section{Alternative Backoff with ECN}

