\chapter{Hosts}

This section describes how each host has been set up from a newly installed \gls{os}.

\section{Preliminaries}

Before you begin, make sure the following have been done:

\begin{itemize}
    \item Install FreeBSD \gls{os} as specified in \ref{install_os}.
    \item Change keyboard layout as specified in \ref{keyboard_layout}.
    \item Create a root user account as specified in \ref{root_account}.
    \item Update the system as specified in \ref{update_system}.
    \item Enable SSH as specified in \ref{enable_ssh}.
\end{itemize}

The rest of the section will assume \lstinline{root} access and that FreeBSD \gls{os} from \ref{install_os} is used.


\section{Network}

The hosts are separated into two subnets. Hosts 2, 3 and 4 are on the network \lstinline{172.16.10.0/24} (VLAN 10), while hosts 5, 6 and 7 are on \lstinline{172.16.20.0/24} (VLAN 20). The main interface on each host is statically connected to the controller network, with an additional virtual interface (or \textit{alias} in FreeBSD) statically connected to the experimental network. To set up each host, run the following, but replace \lstinline{yy} with the appropriate \gls{vlan} and \lstinline{x} with the correct host number as illustrated in \ref{topology}:

\begin{minted}{bash}
# Main interface (controller-network)
echo 'ifconfig_ue0="inet 10.0.1.x netmask 255.255.255.0"' >> /etc/rc.conf
echo 'defaultrouter="10.0.1.254"' >> /etc/rc.conf

# Alias (experiment-network)
echo 'ifconfig_ue0_alias0="inet 172.16.yy.x netmask 255.255.255.0"' >> /etc/rc.conf
\end{minted}

In order for the hosts to communicate between each subnet, a static route must be added. Run the following to do so:

\begin{minted}{bash}
# Static route
echo 'static_routes="intnet"' >> /etc/rc.conf
echo 'route_intnet="-net 172.16.yy.0/24 172.16.xx.1"' >> /etc/rc.conf
\end{minted}

Replace \lstinline{yy} with the destination \gls{vlan} and \lstinline{xx} with the source \gls{vlan}.

Finally, change the hostname to \lstinline{pi3hostX} as specified in \ref{change_hostname} but replace \lstinline{X} with the correct host number, and then reboot to apply all network changes.


\section{NTP Client}

To synchronize time for each host against the gateway, set up each host as an \gls{ntp} client as specified in \ref{time_sync}.


\section{TEACUP}

TEACUP requires four tools that must be installed on each host in order to run experiments, namely \lstinline{lighttpd},  \lstinline{iperf}, \lstinline{httperf} and \lstinline{nttcp}.

\subsubsection{Lighttpd}

The \lstinline{lighttpd} tool can be installed from the package repository with \lstinline{pkg install lighttpd}.

\subsubsection{Iperf}

TEACUP uses a modified version of \lstinline{iperf}, and so must be installed from source as follows:

\begin{minted}{bash}
# Installing modified iperf on FreeBSD
wget https://sourceforge.net/projects/iperf2/files/iperf-2.0.9.tar.gz --no-check-certificate
tar -xf iperf-2.0.9.tar.gz
cd iperf-2.0.9
patch -p1 < iperf-2.0.9.patch
./configure
make
make install
\end{minted}

\subsubsection{Httperf}

TEACUP uses a modified version of \lstinline{httperf}, but installing from source has proven unsuccessful. Thus, an unmodified version is therefore installed with \lstinline{pkg install httperf}.

\subsubsection{Nttcp}

TEACUP uses a modified version of \lstinline{nttcp}, and so must be installed from source as follows:

\begin{minted}{bash}
# Installing modified nttcp on FreeBSD
wget http://caia.swin.edu.au/tools/teacup/downloads/nttcp-1.47-mod.tar.gz
tar -xf nttcp-1.47-mod.tar.gz
cd nttcp-1.47-mod
make
cp nttcp /bin/
\end{minted}