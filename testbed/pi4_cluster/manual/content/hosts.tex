\chapter{Hosts}

This section describes how each host has been set up from a newly installed \gls{os}.

\section{Preliminaries}

Before you begin, make sure the following have been done:

\begin{itemize}
    \item Install an \gls{os} as specified in \ref{install_os}.
    \item Change keyboard layout as specified in \ref{keyboard_layout}.
    \item Create a root user account as specified in \ref{root_account}.
    \item Update the system as specified in \ref{update_system}.
    \item Enable SSH as specified in \ref{enable_ssh}.
\end{itemize}

The rest of the section will assume \lstinline{root} access and that the \gls{os} from \ref{install_os} is used.


\section{Network}

The hosts are separated into two subnets. Hosts 2, 3 and 4 are on the network \lstinline{192.168.10.0/24} (VLAN 10), while hosts 5, 6 and 7 are on \lstinline{192.168.20.0/24} (VLAN 20). The main interface on each host is statically connected to the controller network, with an additional virtual interface statically connected to the experimental network. To set up each host, add the following to the \lstinline{/etc/network/interfaces} file, but replace \lstinline{yy} with the appropriate \gls{vlan} and \lstinline{x} with the correct host number as illustrated in \ref{topology}:

\begin{minted}{bash}
# Main interface (controller-network)
auto eth0
iface eth0 inet static
address 10.0.0.x
netmask 255.255.255.0
gateway 10.0.0.254

# Subinterface (experiment-network)
auto eth0:yy
iface eth0:yy inet static
address 192.168.yy.x
netmask 255.255.255.0
gateway 192.168.yy.1
\end{minted}

In order for the hosts to communicate between each subnet, a static route must be added. Create a file in \lstinline{/etc/network/if-up.d/} called simply \lstinline{route}, and add the following:

\begin{minted}{bash}
#!/bin/sh
ip route add 192.168.yy.0/24 via 192.168.xx.1
\end{minted}

Replace \lstinline{yy} with the destination \gls{vlan} and \lstinline{xx} with the source \gls{vlan}. For this script to be executed upon reboot, make it executable with the command \lstinline{chmod +x /etc/network/if-up.d/route}.

Finally, change the hostname to \lstinline{pi4hostX} as specified in \ref{change_hostname} but replace \lstinline{X} with the correct host number, and then reboot to apply all network changes.


\section{NTP Client}

To synchronize time for each host against the gateway, set up each host as an \gls{ntp} client as specified in \ref{time_sync}.

\section{Web10g}

\todo{show how to compile and install web10g kernel for Pi4}


\section{TEACUP}

TEACUP requires four tools that must be installed on each host in order to run experiments, namely \lstinline{lighttpd},  \lstinline{iperf}, \lstinline{httperf} and \lstinline{nttcp}. \todo{fix web10g first...}