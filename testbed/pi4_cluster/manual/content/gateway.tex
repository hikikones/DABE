\chapter{Gateway}

\todo{briefly explain section content}

\section{Preliminaries}

Before you begin, make sure the following have been done:

\begin{itemize}
    \item Install an \gls{os} as specified in \ref{install_os}.
    \item Change keyboard layout as specified in \ref{keyboard_layout}.
    \item Create a root user account as specified in \ref{root_account}.
    \item Update the system as specified in \ref{update_system}.
    \item Enable SSH as specified in \ref{enable_ssh}.
\end{itemize}

The rest of the section will assume \lstinline{root} access and that the \gls{os} from \ref{install_os} is used.


\section{Network}

The gateway is the only machine with direct Internet access through \gls{dhcp} on its main interface, with a virtual interface statically connected to the private controller network in order to communicate with the rest of the machines. To set up this, add the following to the \lstinline{/etc/network/interfaces} file:

\begin{lstlisting}
# Main interface (Internet access)
auto eth0
iface eth0 inet dhcp

# Subinterface (controller-network)
auto eth0:1
iface eth0:1 inet static
address 10.10.10.254
netmask 255.255.255.0
\end{lstlisting}

In order for the machines on the internal network to gain Internet access through the gateway, both IP forwarding and \gls{nat} must be set up:

\begin{lstlisting}
# Enable IP forwarding
echo 'net.ipv4.ip_forward=1' >> /etc/sysctl.conf

# NAT
update-alternatives --set iptables /usr/sbin/iptables-legacy
iptables -t nat -A POSTROUTING -o eth0 -j MASQUERADE
apt install iptables-persistent
\end{lstlisting}

Finally, change the hostname to \lstinline{pigate} as specified in \ref{change_hostname} and reboot to apply all network changes.

\todo{ssh login without password}


\section{TEACUP}

TEACUP requires a bunch of tools in order to function properly, as noted from the \href{http://caia.swin.edu.au/tools/teacup/TEACUP-0.9_INSTALL.txt}{install} guide. The following commands will install TEACUP properly:

\begin{lstlisting}
# R
apt install -y r-base

# PDFJAM
apt install -y texlive-extra-utils

# SPP
apt install -y mercurial libpcap-dev build-essential
hg clone https://bitbucket.org/caia-swin/spp
cd spp
make
mkdir /usr/local/man/man1
make install
cd ..

# Fabric
apt install -y fabric python-pip python-dev libffi-dev libssl-dev
pip install fabric3
pip install -Iv pexpect==3.2

# TEACUP
wget https://sourceforge.net/projects/teacup/files/teacup-1.1.tar.gz
tar -xf teacup-1.1.tar.gz
\end{lstlisting}

To verify that TEACUP has been properly installed, a simple check can be done as follows:

\begin{lstlisting}
mkdir experiment
cp teacup-1.1/example_configs/config-scenario1.py experiment/config.py
cp teacup-1.1/run.sh experiment/
cp teacup-1.1/fabfile.py experiment/
\end{lstlisting}

Go into the \lstinline{experiment} folder and prepare to edit the \lstinline{config.py} file. Find the line containing \lstinline{TPCONF_script_path = '/home/teacup/teacup-0.8'} and change the \lstinline{path} to where you extracted \lstinline{teacup-1.1}. Finally, verify TEACUP installation with the command \lstinline{fab check_config}. If no errors appear, all is good.


\subsection{Modification}

Due to the restricted physical set up as explained in \ref{pi4cluster} and \ref{topology}, virtual interfaces have been used extensively. \todo{explain why tcpdump doesn't like this}

A slight adjustment to TEACUP was therefore necessary.

\todo{show the teacup tcpdump unique name modification}

\todo{completed experiments must be analysed on an x86 computer since the tool spp shits the bed on ARM}