\chapter{Gateway}

This section describes how the machine with direct Internet access has been set up, known as the \textit{gateway}, and how it provides access and Internet for the other machines through \gls{nat}. The gateway is also known as the \textit{controller}, as this is where TEACUP experiments are both configured and conducted from.

\section{Preliminaries}

Before you begin, make sure the following have been done:

\begin{itemize}
    \item Install an \gls{os} as specified in \ref{install_os}.
    \item Change keyboard layout as specified in \ref{keyboard_layout}.
    \item Create a root user account as specified in \ref{root_account}.
    \item Update the system as specified in \ref{update_system}.
    \item Enable SSH as specified in \ref{enable_ssh}.
\end{itemize}

The rest of the section will assume \lstinline{root} access and that the \gls{os} from \ref{install_os} is used.


\section{Network}

The gateway is the only machine with direct Internet access through \gls{dhcp} on its main interface, with a virtual interface statically connected to the private controller network in order to communicate with the rest of the machines. To set up this, add the following to the \lstinline{/etc/network/interfaces} file:

\begin{minted}{bash}
# Main interface (Internet access)
auto eth0
iface eth0 inet dhcp

# Subinterface (controller-network)
auto eth0:1
iface eth0:1 inet static
address 10.0.0.254
netmask 255.255.255.0
\end{minted}

In order for the machines on the internal network to gain Internet access through the gateway, both IP forwarding and \gls{nat} must be set up:

\begin{minted}{bash}
# Enable IP forwarding
echo 'net.ipv4.ip_forward=1' >> /etc/sysctl.conf

# NAT
update-alternatives --set iptables /usr/sbin/iptables-legacy
iptables -t nat -A POSTROUTING -o eth0 -j MASQUERADE
apt install iptables-persistent
\end{minted}

Finally, change the hostname to \lstinline{pi4gate} as specified in \ref{change_hostname} and reboot to apply all network changes.


\section{NTP Server}

TEACUP requires that time is synchronized on all machines when running experiments. In order for the internal machines to synchronize their clock, an \gls{ntp} server must be set up. The gateway will provide this service, and is set up as specified in \ref{time_sync}.


\section{TEACUP} \label{teacup_gateway}

Every network experiment orchestrated by TEACUP is initiated from the gateway, and so the majority of the tools for TEACUP to work are installed on this machine. As described from the official \href{http://caia.swin.edu.au/tools/teacup/TEACUP-0.9_INSTALL.txt}{install} guide, with some slight adjustments, the following commands will install TEACUP properly:

\begin{minted}{bash}
# R
apt install -y r-base

# PDFJAM
apt install -y texlive-extra-utils

# SPP
apt install -y mercurial libpcap-dev build-essential
hg clone https://bitbucket.org/caia-swin/spp
cd spp
make
mkdir /usr/local/man/man1
make install
cd ..

# Fabric
apt install -y fabric python-pip python-dev libffi-dev libssl-dev
pip install fabric3
pip install -Iv pexpect==3.2

# TEACUP
wget https://sourceforge.net/projects/teacup/files/teacup-1.1.tar.gz
tar -xf teacup-1.1.tar.gz
\end{minted}

To verify that TEACUP has been properly installed, a simple check can be done as follows:

\begin{minted}{bash}
mkdir experiment
cp teacup-1.1/example_configs/config-scenario1.py experiment/config.py
cp teacup-1.1/run.sh experiment/
cp teacup-1.1/fabfile.py experiment/
\end{minted}

Go into the \lstinline{experiment} folder and prepare to edit the \lstinline{config.py} file. Find the line containing \lstinline{TPCONF_script_path = '/home/teacup/teacup-0.8'} and change the \lstinline{path} to where you extracted \lstinline{teacup-1.1}. Finally, verify TEACUP installation with the command \lstinline{fab check_config}. If no errors appear, all is good.


\subsection{Modifications}

Due to the restricted physical set up as shown in \ref{fig:pi4cluster} and discussed in \ref{topology}, virtual interfaces have been used extensively, particularly on the router. This introduced a conflict to the TEACUP codebase, as it assigns a name to each interface before performing network logging with \lstinline{tcpdump}. However, this name is not unique since both the controller and experiment network shares the same interface, resulting in a \lstinline{duplicate handle error} from TEACUP. A slight adjustment to the codebase was therefore necessary.

Applying the following "band-aid" to the \lstinline{loggers.py} file in TEACUP should fix the error:

\begin{minted}{bash}
--- teacup-1.1/loggers.py
+++ teacup-1.1-modified/loggers.py
@@ -41,6 +41,7 @@
    from getfile import getfile
    from runbg import runbg
    
+   import random
    
    ## Collect all the arguments (here basically a dummy method because we
    ## dont used the return value)
@@ -505,7 +506,7 @@
                    snap_len, interface, file_name, tcpdump_filter)
        pid = runbg(tcpdump_cmd)
    
-       name = 'tcpdump-' + interface
+       name = 'tcpdump-' + interface + str(random.randint(0, 50000))
        #bgproc.register_proc(env.host_string, name, '0', pid, file_name)
        bgproc.register_proc_later(
            env.host_string,
\end{minted}

\todo{the following modification only applies to FreeBSD... PI4-cluster uses only Linux, but maybe it is also applies there?}

In additon, a few changes to the \lstinline{hostsetup.py} file in TEACUP must be done. TEACUP checks if the \lstinline{sysctl} variable \lstinline{net.inet.tcp.reass.overflows} exists, which it does not for the FreeBSD version that we are using. The default sender and receiver buffer on FreeBSD hosts must also be set to a higher value, otherwise TEACUP produces results with a capped \gls{cwnd} value. The changes are easily done as follows:

\begin{minted}{bash}
--- teacup-1.1/hostsetup.py
+++ teacup-1.1-modified/hostsetup.py
@@ -858,7 +858,7 @@
 
     if htype == 'FreeBSD':
         # record the number of reassembly queue overflows
-        run('sysctl net.inet.tcp.reass.overflows')
+        #run('sysctl net.inet.tcp.reass.overflows')
 
         # disable auto-tuning of receive buffer
         run('sysctl net.inet.tcp.recvbuf_auto=0')
@@ -869,6 +869,8 @@
         # send and receiver buffer max (2MB by default on FreeBSD 9.2 anyway)
         run('sysctl net.inet.tcp.sendbuf_max=2097152')
         run('sysctl net.inet.tcp.recvbuf_max=2097152')
+        run('sysctl net.inet.tcp.recvspace=655360')
+        run('sysctl net.inet.tcp.sendspace=327680')
 
         # clear host cache quickly, otherwise successive TCP connections will
         # start with ssthresh and cwnd from the end of most recent tcp
\end{minted}

Finally, the tool \lstinline{spp} does not run on the Raspberry Pi 4 or ARM architecture at all, only displaying the helpful message "\lstinline{aborting}" when running it. TEACUP uses this tool to calculate \gls{rtt}, and fortunately is only needed when analyzing the results. In other words, the experiment data can easily be copied over to an x86 machine where \lstinline{spp} works, and then be analyzed from there. That is, follow the install instructions above from \ref{teacup_gateway} again as normal, but on an x86 machine instead.