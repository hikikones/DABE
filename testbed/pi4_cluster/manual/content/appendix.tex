\chapter{General}

\section{Installing an Operating System} \label{install_os}

To use the Raspberry Pi, an \gls{os} first needs to installed. \href{https://www.raspberrypi.org/downloads/raspbian/}{Raspbian Buster Lite} is the official \gls{os} for Raspberry Pi 4, and is used on every machine in the cluster.


After downloading the \gls{os}, uncompress the file. The extracted file should be of type \lstinline{.img}, containing a preinstalled Raspbian Buster \gls{os} that now must be written to the SD card. To write the image file to the SD card, several tools exists that can be used, with some examples following:

\begin{itemize}
    \item balenaEtcher (cross-platform)
    \item \lstinline{dd} (Linux)
\end{itemize}

Once the image has been successfully written, simply put the SD card into the Raspberry Pi machine and boot it. The default user that comes preinstalled is called \lstinline{pi}, with the default password \lstinline{raspberry}.


\section{Keyboard Layout} \label{keyboard_layout}

To change keyboard layout with a graphical user guide, simply run \lstinline{dpkg-reconfigure keyboard-configuration}. For most full-sized keyboards, the following options are what you want; \lstinline{Generic 105-key PC (intl.) -> Norwegian -> Norwegian -> The default for the keyboard layout -> No compose key}.


\section{Root Account} \label{root_account}

Once logged in with user \lstinline{pi}, create a root user with \lstinline{sudo passwd root}. Enter a chosen password. A root user has now been created. To use it, log out with \lstinline{logout} or simply reboot, and then log in as \lstinline{root}.

To delete the \lstinline{pi} user, issue the command \lstinline{deluser --remove-home pi} while logged in as \lstinline{root}.


\section{Updating the System} \label{update_system}

\textbf{Note:} \textit{If the current system is using a custom kernel, upgrading the system will also upgrade the kernel, thus overwriting the custom kernel. One can omit the \lstinline{upgrade} part if that is not desired.}

To update the system, run \lstinline{apt update && apt upgrade}. If an error about \lstinline{release file not valid yet} appears, the system's clock needs to be fixed. This can easily be done with the \lstinline{date} tool; \lstinline{date -s "15 Feb 2020 12:00"}.


\section{Enable SSH} \label{enable_ssh}

To enable SSH access, simply run \lstinline{systemctl enable ssh}. To allow SSH root login, the line \lstinline{PermitRootLogin yes} must be added to the file \lstinline{/etc/ssh/sshd_config}, which can conveniently be done with \lstinline{echo 'PermitRootLogin yes' >> /etc/ssh/sshd_config}. Then start SSH service with \lstinline{systemctl start ssh} or just reboot.


\section{Change Hostname} \label{change_hostname}

There are two files that needs to be edited in order to change hostname. First, the \lstinline{/etc/hostname} file, which only contains the current hostname. Simply change whatever is in it to what you want, say \lstinline{new_hostname}. Second, the file \lstinline{/etc/hosts} needs one line changed. Edit the line containg \lstinline{127.0.1.1} so that it looks as follows:

\begin{lstlisting}
127.0.1.1       new_hostname
\end{lstlisting}

Reboot to apply the changes.


\section{Time Synchronization} \label{time_sync}

First, set up a common timezone on all machines with \lstinline{timedatectl set-timezone CET}.

\subsubsection{NTP Server}

Install \lstinline{ntp} with \lstinline{apt install ntp}. Then check if any \gls{ntp} peers are connected with \lstinline{ntpq -p}. If not, consider replacing the default pools or servers in \lstinline{/etc/ntp.conf} with another, such as \lstinline{server ntp.uio.no} followed up by \lstinline{systemctl restart ntp}.

\subsubsection{NTP Client}

To automatically query for time from the \gls{ntp} server, the Raspbian Buster \gls{os} already comes with a lightweight daemon called \lstinline{systemd-timesyncd} that allows for synchronizing the system clock across the network. However, TEACUP prefers to use \lstinline{ntp} instead, and so \lstinline{ntp} will be used.

Install \lstinline{ntp} with \lstinline{apt install ntp}. To specify which server to get the time from, prepare to edit the \lstinline{/etc/ntp.conf} file. Comment out the default pools, and add the entry \lstinline{server 10.0.0.254 iburst} which refers to the gateway.

Next, disable the \lstinline{systemd-timesyncd} with \lstinline{timedatectl set-ntp false} as we are using \lstinline{ntp} directly to update the time. Finally, either restart the service with \lstinline{systemctl restart ntp} or reboot. Wait a bit, and verify time synchronization with either \lstinline{ntpq -p} or simply \lstinline{date}.