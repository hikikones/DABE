\newglossaryentry{tcp}
{
    name={Transmission Control Protocol (TCP)},
    first={Transmission Control Protocol (TCP)},
    text={TCP},
    description={One of the main communication protocols of the Internet that defines how to establish and maintain a network conversation through which applications can exchange data.}
}

\newglossaryentry{www}
{
    name={World Wide Web (WWW)},
    first={World Wide Web (WWW)},
    text={WWW},
    description={Commonly known as the Web; an information system in the form of webpages that web browsers can read and interact with.}
}

\newglossaryentry{aimd}
{
    name={Additive-increase/Multiplicative-decrease (AIMD)},
    first={additive-increase/multiplicative-decrease (AIMD)},
    text={AIMD},
    description={A feedback control algorithm best known for its use in TCP congestion control. AIMD combines linear growth of the congestion window with an exponential reduction when congestion is detected.}
}

\newglossaryentry{cwnd}
{
    name={Congestion Window (CWND)},
    first={Congestion Window (CWND)},
    text={CWND},
    description={A TCP state variable that limits the amount of data the sender can send into the network before receiving an ACK.}
}

\newglossaryentry{rwnd}
{
    name={Receiver Window (RWND)},
    first={Receiver Window (RWND)},
    text={RWND},
    description={A TCP state variable that advertises the amount of data that the receiver can receive.}
}

\newglossaryentry{mss}
{
    name={Maximum Segment Size (MSS)},
    first={Maximum Segment Size (MSS)},
    text={MSS},
    description={The largest specified amount of data (in bytes) that a communications device can receive in a single TCP segment.}
}

\newglossaryentry{rtt}
{
    name={Round Trip Time (RTT)},
    first={Round Trip Time (RTT)},
    text={RTT},
    description={The time it takes for a signal to be sent plus the time for an ACK of that signal to be received.}
}

\newglossaryentry{ack}
{
    name={Acknowledgement (ACK)},
    first={acknowledgement (ACK)},
    text={ACK},
    description={A signal that is passed between communicating processes, computers, or devices to signify acknowledgement, or receipt of message, as part of a communications protocol.}
}

\newglossaryentry{cc}
{
    name={Congestion Control (CC)},
    first={Congestion Control (CC)},
    text={CC},
    description={The process of managing the sender's packet rate to not overwhelm the network.}
}

\newglossaryentry{ssthresh}
{
    name={Slow start threshold (ssthresh)},
    first={slow start threshold (ssthresh)},
    text={ssthresh},
    description={A TCP state variable used to determine whether the slow start or congestion avoidance algorithm is used to control data transmission.}
}

\newglossaryentry{ecn}
{
    name={Explicit Congestion Notification (ECN)},
    first={Explicit Congestion Notification (ECN)},
    text={ECN},
    description={An extension to IP and TCP that allows end-to-end notification of network congestion without dropping packets.}
}

\newglossaryentry{grub}
{
    name={GRand Unified Bootloader (GRUB)},
    first={GRand Unified Bootloader (GRUB)},
    text={GRUB},
    description={A Multiboot boot loader. It was derived from GRUB, the GRand Unified Bootloader, which was originally designed and implemented by Erich Stefan Boleyn.}
}

\newglossaryentry{os}
{
    name={Operating System (OS)},
    first={Operating System (OS)},
    text={OS},
    description={System software that manages computer hardware, software resources, and provides common services for computer programs.}
}

\newglossaryentry{nat}
{
    name={Network Address Translation (NAT)},
    first={Network Address Translation (NAT)},
    text={NAT},
    description={A method of remapping one IP address space into another. Often used such that one Internet-routable IP address of a NAT gateway can be used for an entire private network. This is used in conjunction with IP masquerading, which is a technique that hides an entire IP address space, usually consisting of private IP addresses, behind a single IP address in another, usually public address space.}
}

\newglossaryentry{dhcp}
{
    name={Dynamic Host Configuration Protocol (DHCP)},
    first={Dynamic Host Configuration Protocol (DHCP)},
    text={DHCP},
    description={A client/server protocol that automatically provides an Internet Protocol (IP) host with its IP address and other related configuration information such as the subnet mask and default gateway.}
}

\newglossaryentry{vlan}
{
    name={Virtual LAN (VLAN)},
    first={virtual LAN (VLAN)},
    text={VLAN},
    description={Any broadcast domain that is partitioned and isolated in a computer network at the data link layer (OSI layer 2).}
}

\newglossaryentry{ntp}
{
    name={Network Time Protocol (NTP)},
    first={Network Time Protocol (NTP)},
    text={NTP},
    description={A networking protocol for clock synchronization between computer systems.}
}

\newglossaryentry{poe}
{
    name={Power over Ethernet (PoE)},
    first={Power over Ethernet (PoE)},
    text={PoE},
    description={A standard or ad hoc systems which pass electric power along with data on twisted pair Ethernet cabling. This allows a single cable to provide both data connection and electric power to devices such as wireless access points, IP cameras, VoIP phones and later Raspberry Pi machines.}
}